\documentclass{article}
%%%%%%%%%%%%%%%%%%%%%%%%%%%%% Define Article %%%%%%%%%%%%%%%%%%%%%%%%%%%%%%%%%%
%%%%%%%%%%%%%%%%%%%%%%%%%%%%%%%%%%%%%%%%%%%%%%%%%%%%%%%%%%%%%%%%%%%%%%%%%%%%%%%

%%%%%%%%%%%%%%%%%%%%%%%%%%%%% Using Packages %%%%%%%%%%%%%%%%%%%%%%%%%%%%%%%%%%
\usepackage{float}
\usepackage[letterpaper,portrait]{geometry}
\usepackage{graphicx}
\usepackage{anysize}
\usepackage{lipsum}
\usepackage{amsmath,amssymb,amsthm}
\usepackage[utf8]{inputenc}
\usepackage{multirow}
\usepackage{csquotes}
\usepackage[spanish]{babel}
\usepackage{apacite}
\usepackage{multicol}
\usepackage{parskip}
\usepackage{setspace}
\usepackage{empheq}
\usepackage{mdframed}
\usepackage{booktabs}
\usepackage{lipsum}
\usepackage{graphicx}
\usepackage{color}
\usepackage{psfrag}
\usepackage{pgfplots}
\usepackage{bm}
\usepackage{tocloft}

%%%%%%%%%%%%%%%%%%%%%%%%%%%%%%%%%%%%%%%%%%%%%%%%%%%%%%%%%%%%%%%%%%%%%%%%%%%%%%%

% Other Settings

%%%%%%%%%%%%%%%%%%%%%%%%%% Page Setting %%%%%%%%%%%%%%%%%%%%%%%%%%%%%%%%%%%%%%%
\geometry{letterpaper, margin=2.54cm}

%%%%%%%%%%%%%%%%%%%%%%%%%% Define some useful colors %%%%%%%%%%%%%%%%%%%%%%%%%%
\definecolor{ocre}{RGB}{243,102,25}
\definecolor{mygray}{RGB}{243,243,244}
\definecolor{deepGreen}{RGB}{26,111,0}
\definecolor{shallowGreen}{RGB}{235,255,255}
\definecolor{deepBlue}{RGB}{61,124,222}
\definecolor{shallowBlue}{RGB}{235,249,255}
%%%%%%%%%%%%%%%%%%%%%%%%%%%%%%%%%%%%%%%%%%%%%%%%%%%%%%%%%%%%%%%%%%%%%%%%%%%%%%%

%%%%%%%%%%%%%%%%%%%%%%%%%% Define an orangebox command %%%%%%%%%%%%%%%%%%%%%%%%
\newcommand\orangebox[1]{\fcolorbox{ocre}{mygray}{\hspace{1em}#1\hspace{1em}}}
%%%%%%%%%%%%%%%%%%%%%%%%%%%%%%%%%%%%%%%%%%%%%%%%%%%%%%%%%%%%%%%%%%%%%%%%%%%%%%%

%%%%%%%%%%%%%%%%%%%%%%%%%%%% English Environments %%%%%%%%%%%%%%%%%%%%%%%%%%%%%
\newtheoremstyle{mytheoremstyle}{3pt}{3pt}{\normalfont}{0cm}{\rmfamily\bfseries}{}{1em}{{\color{black}\thmname{#1}~\thmnumber{#2}}\thmnote{\,--\,#3}}
\newtheoremstyle{myproblemstyle}{3pt}{3pt}{\normalfont}{0cm}{\rmfamily\bfseries}{}{1em}{{\color{black}\thmname{#1}~\thmnumber{#2}}\thmnote{\,--\,#3}}
\theoremstyle{mytheoremstyle}
\newmdtheoremenv[linewidth=1pt,backgroundcolor=shallowGreen,linecolor=deepGreen,leftmargin=0pt,innerleftmargin=20pt,innerrightmargin=20pt,]{theorem}{Theorem}[section]
\theoremstyle{mytheoremstyle}
\newmdtheoremenv[linewidth=1pt,backgroundcolor=shallowBlue,linecolor=deepBlue,leftmargin=0pt,innerleftmargin=20pt,innerrightmargin=20pt,]{definition}{Definition}[section]
\theoremstyle{myproblemstyle}
\newmdtheoremenv[linecolor=black,leftmargin=0pt,innerleftmargin=10pt,innerrightmargin=10pt,]{problem}{Problem}[section]
%%%%%%%%%%%%%%%%%%%%%%%%%%%%%%%%%%%%%%%%%%%%%%%%%%%%%%%%%%%%%%%%%%%%%%%%%%%%%%%

%%%%%%%%%%%%%%%%%%%%%%%%%%%%%%% Plotting Settings %%%%%%%%%%%%%%%%%%%%%%%%%%%%%
\usepgfplotslibrary{colorbrewer}
\pgfplotsset{width=8cm,compat=1.9}
%%%%%%%%%%%%%%%%%%%%%%%%%%%%%%%%%%%%%%%%%%%%%%%%%%%%%%%%%%%%%%%%%%%%%%%%%%%%%%%

%%%%%%%%%%%%%%%%%%%%%%%%%%%%%%% Title & Author %%%%%%%%%%%%%%%%%%%%%%%%%%%%%%%%
\author{Gustavo Vergara}
%%%%%%%%%%%%%%%%%%%%%%%%%%%%%%%%%%%%%%%%%%%%%%%%%%%%%%%%%%%%%%%%%%%%%%%%%%%%%%%


\begin{document}
\pgfplotsset{compat=1.18}
\setstretch{2}

\begin{titlepage}
    \centering
    \vspace{2.5cm}
    {\scshape \Large Taller sobre metodologías de desarrollo de software - GA1-220501093-AA1-EV01 \par}
    \vspace{5cm}
    \textbf\large\scshape{\par}
         \vspace{0.5cm}
         
    {\Large Vergara Pareja Gustavo\par}
    \vspace{5cm}
    {\scshape\Large Jovanna Herazo\par}
    \vspace{0.3cm}
    {\scshape\Large Tecnología en Análisis y Desarrollo de Software \par}
    \vspace{0.3cm}
    {\scshape\Large SENA - Centro Agropecuario Regional Cauca\par}
    \vspace{0.3cm}
    {\Large \today \par}
    \end{titlepage}

\begin{flushleft}
    \large \textbf{EVIDENCIA A SOLUCIONAR}\\
    \vspace{0.1cm}
    \section*{Evidencia conocimiento: GA1-220501093-AA1-EV01 taller sobre metodologías de desarrollo de software}
    Las metodologías de desarrollo son indispensables en los grupos de trabajo y organizaciones relacionadas con la industria de software, partiendo de la información abordada en este componente desarrollar el taller sobre metodologías de desarrollo de software propuesto.
    
Elementos para tener en cuenta en el taller:
\begin{itemize}
    \item Seleccionar diferentes fuentes de información relacionadas con las metodologías de desarrollo de software.
    \item Detallar las características que identifican a los marcos de trabajo tradicionales y los marcos de trabajo
    ágiles.
    \item Utilizar imágenes de construcción propia o que tengan los derechos respectivos de uso.
\end{itemize}   
    \end{flushleft}
    \newpage
    %\tableofcontents
%\section{Ecuación del área total}
%\begin{figure}[H]
   % \centering
    %\includegraphics[width=0.6\textwidth]{CASA3.png}
    %\caption{Casa de chocolate}
    %\label{fig:imagen1}
   % \end{figure}

%\begin{figure}[H]
    %\centering
   % \includegraphics[width=0.6\textwidth]{2.png}
   % \caption{Áreas de la casa}
   % \label{fig:imagen1}
   % \end{figure}
%Sabiendo que b y h y l, son distintos para cada área, entonces la ecuación que representa el área total será:
%\begin{figure}[H]
   % \centering
   % \includegraphics[width=0.6\textwidth]{1.png}
    
    %\label{fig:imagen1}
    %\end{figure}
\section{FORMULACIÓN DE PREGUNTAS}
De acuerdo con las temáticas desarrolladas en el componente formativo resolver las siguientes inquietudes.
\newline 1. Describa con sus propias palabras qué es y de que se compone una metodología de desarrollo de software. Citar por lo menos 2 datos que demuestren su utilidad (35\%).
\newline 
Una metodología de desarrollo de software es un conjunto de enfoques, prácticas y procesos estructurados que se utilizan para gestionar y controlar el desarrollo de software. Estas metodologías proporcionan un marco de trabajo para planificar, diseñar, implementar y mantener software de manera eficiente y efectiva.
\newline
Dos datos que demuestran la utilidad de una metodología de desarrollo de software son:
\begin{itemize}
    \item Mejora la productividad y calidad del software: Una metodología bien definida y seguida adecuadamente puede ayudar a mejorar la productividad del equipo de desarrollo, al proporcionar pautas claras sobre cómo llevar a cabo las diferentes etapas del proceso de desarrollo. Esto también contribuye a la calidad del software final entregado, ya que se pueden aplicar prácticas de control de calidad y pruebas en cada etapa.
    \item Facilita la gestión y el seguimiento del proyecto: Una metodología de desarrollo de software define roles y responsabilidades claras para los miembros del equipo, y establece hitos y plazos para el progreso del proyecto. Esto facilita la gestión del proyecto, ya que se pueden identificar y abordar problemas potenciales de manera temprana, lo que ayuda a evitar retrasos y desviaciones.
\end{itemize}
 2. Describa con sus propias palabras cuáles son las características fundamentales de un marco de trabajo ágil y un marco de trabajo tradicional (35\%).
\newline Características de un marco de trabajo ágil y un marco de trabajo tradicional:
\newline Marco de trabajo ágil: Un marco de trabajo ágil se caracteriza por su enfoque iterativo e incremental en el desarrollo de software. Algunas características fundamentales de un marco ágil son:
\begin{itemize}
    \item Flexibilidad y adaptabilidad: Los marcos ágiles se adaptan a los cambios y requisitos emergentes durante el desarrollo del proyecto. Se enfocan en la colaboración y la comunicación continua con los stakeholders para garantizar la satisfacción del cliente.
    Entrega temprana y frecuente: Los marcos ágiles priorizan la entrega de incrementos de software funcionales de forma temprana y regular. Esto permite obtener retroalimentación rápida y realizar ajustes según sea necesario. 
    \end{itemize}

Marco de trabajo tradicional: Un marco de trabajo tradicional, a menudo conocido como enfoque de desarrollo en cascada, se caracteriza por un enfoque secuencial y planificado del desarrollo de software. Algunas características fundamentales de un marco de trabajo tradicional son:
\begin{itemize}
    \item Planificación y documentación exhaustiva: Los marcos tradicionales enfatizan la planificación y la documentación detallada antes de que comience la implementación del software. Se espera tener una comprensión clara y completa de los requisitos antes de pasar a la siguiente fase.
    \item Enfoque secuencial: Los marcos tradicionales siguen una secuencia lineal de etapas, como el análisis de requisitos, el diseño, la implementación y las pruebas. Cada etapa se completa antes de pasar a la siguiente.
\end{itemize}
3. Elabore una lista donde clasifique por lo menos cinco metodologías de desarrollo de software en marcos tradicionales y marcos ágiles (30\%).

Metodologías en marcos tradicionales:
\begin{itemize}
    \item Modelo en cascada (Waterfall)
    \item Modelo en V (V-Model)
    \item Modelo en espiral (Spiral Model)
\end{itemize}

Metodologías en marcos ágiles:
\begin{itemize}
    \item Scrum
    \item Extreme Programming (XP)
    \item Kanban
    \item Lean Software Development
    \item Desarrollo de software adaptativo (Adaptive Software Development)
\end{itemize}
\begin{figure}
    \centering
    \includegraphics[width=0.6\textwidth]{132_1584681-W.jpg}
    \caption{Software development}
\end{figure}

\section{Referencias}
Encyclopædia Britannica. (n.d.). Computer programmers discuss software development. [Photograph]. Britannica ImageQuest. Retrieved October 19, 2023, from https://quest-eb-com.bdigital.sena.edu.co/images/132_1584681
%\bibliographystyle{apacite}
%\nocite{*}
%\bibliography{referenciados}
\end{document}